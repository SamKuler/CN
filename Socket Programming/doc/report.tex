\documentclass[11pt,a4paper]{article}
\usepackage[UTF8]{ctex}
\usepackage[margin=1.5cm]{geometry}
\usepackage{listings}
\usepackage{xcolor}
\usepackage{enumitem}
\usepackage{booktabs}
\usepackage{hyperref}
\usepackage{multicol}

% 代码样式设置
\lstset{
    basicstyle=\ttfamily\small,
    breaklines=true,
    frame=single,
    backgroundcolor=\color{gray!10}
}

% 紧凑列表
\setlist{nosep, leftmargin=*}

% 紧凑段落
\setlength{\parskip}{2pt}
\setlength{\parindent}{0pt}

\title{\textbf{基于Socket的FTP服务器与客户端实现}}
\author{Zeng Guanyang}
\date{}

\begin{document}

\maketitle
\vspace{-2.5em}

\section{简介}

本项目实现了完整的FTP系统,包含多线程C语言服务器和Python客户端。
服务器支持并发连接、异步传输、主被动模式、断点续传、文件锁等功能,可在Linux/Windows平台运行。
客户端提供CLI和GUI界面,支持同步和异步传输。
实现了所有Optional功能。

代码量:\textbf{服务器}:约\textbf{13000}行;\textbf{客户端}:约\textbf{4000}行。

\section{核心功能与支持的FTP命令}

\subsection{服务器端实现(C语言)}

\textbf{模块设计:}
\begin{itemize}[itemsep=0pt]
    \item \textbf{server.c:}服务器核心,负责监听套接字管理、连接上限控制、线程池和会话生命周期管理
    \item \textbf{session.c:}会话状态管理,维护认证状态、当前目录、数据连接模式、续传偏移、重命名状态、统计信息等
    \item \textbf{command.c + handler.c:}命令注册与调度,由 handler 实现所有FTP命令的具体逻辑和权限校验
    \item \textbf{auth.c:}用户认证模块,加载users.db(可选),支持密码哈希校验(简单实现)和权限管理
    \item \textbf{filesys.c:}跨平台文件系统抽象层,提供路径操作、目录遍历、文件元信息查询等功能
    \item \textbf{transfer.c:}数据传输引擎,管理数据通道、支持ASCII/Binary模式、断点续传、传输中断等
    \item \textbf{filelock.c:}文件锁管理器,实现基于路径的读写锁,保证并发访问安全性
    \item \textbf{network.c:}网络抽象层,统一 POSIX 和 Winsock API,提供简单的超时控制、非阻塞模式等
    \item \textbf{logger.c:}日志系统,支持分级(彩色)输出、时间戳、文件名/行号/函数名记录
\end{itemize}

\textbf{支持的FTP命令:}

\begin{multicols}{2}
\begin{itemize}[itemsep=0pt]
    \item \textbf{连接管理:}USER, PASS, QUIT, REIN
    \item \textbf{传输模式:}PORT, PASV, TYPE, MODE, STRU
    \item \textbf{文件操作:}RETR, STOR, APPE, DELE, REST
    \item \textbf{目录操作:}LIST, NLST, CWD, CDUP, PWD, MKD, RMD
    \item \textbf{重命名:}RNFR, RNTO
    \item \textbf{传输控制:}ABOR
    \item \textbf{信息查询:}SYST, FEAT, SIZE, MDTM, NOOP
\end{itemize}
\end{multicols}

\textbf{关键特性:}
\begin{itemize}[itemsep=1pt]
    \item \textbf{用户认证系统:}支持多用户管理(users.db),密码哈希校验,权限控制(读/写/删除/重命名/目录管理/管理员),匿名用户为兼容作业具有管理员权限
    \item \textbf{安全隔离:}所有路径相对配置根目录进行归一化验证,确保用户无法访问根目录外的文件
    \item \textbf{并发文件锁:}基于路径的读写锁机制,以支持多客户端同时下载,写操作互斥,防止竞态条件
    \item \textbf{断点续传:}REST命令配合RETR/STOR实现从指定偏移量继续传输
    \item \textbf{中断控制:}ABOR命令支持通过 OOB 即时中断正在进行的传输
    \item \textbf{会话统计:}记录上传/下载字节数、文件数量、命令总数和在线时长
\end{itemize}

\subsection{客户端实现(Python)}

\textbf{架构:}FTPClient主控,ControlConnection/DataConnection管理连接,CommandRegistry实现命令注册模式,TransferManager管理异步传输。实现CLI和Tkinter GUI界面。

\textbf{命令:}USER, PASS, PASV, PORT, RETR, STOR, APPE, REST, ABOR, LIST, NLST, CWD, CDUP, PWD, MKD, RMD, DELE, RNFR, RNTO, TYPE, QUIT等

\textbf{特性:}同步/异步传输、GUI断点续传、进度显示、CLI便捷命令(upload, download, ls等)

\section{关键实现细节}

\subsection{服务器端核心实现}

\textbf{1. 多线程会话管理}

服务器在\texttt{server.c}的\texttt{client\_thread()}函数中处理每个客户端连接。主要流程:
\begin{itemize}[itemsep=0pt]
    \item 通过\texttt{pthread\_mutex}保护的全局计数器\texttt{g\_current\_connections}实现并发连接数限制
    \item 调用\texttt{net\_set\_oob\_inline()}配置控制套接字接收带外数据(用于ABOR命令)
    \item 发送220欢迎消息后,进入命令循环:使用\texttt{net\_receive\_line()}接收命令,\texttt{net\_has\_urgent\_data()}检测紧急数据(但实际和正常信息流统一处理)
    \item 命令解析后通过\texttt{command\_dispatch()}分发到对应的handler函数执行
    \item 会话结束时调用\texttt{session\_destroy()},等待传输线程退出并递减连接计数
\end{itemize}

\textbf{2. 异步数据传输线程}

handler函数调用\texttt{session\_start\_transfer\_thread()}启动传输线程:
\begin{itemize}[itemsep=0pt]
    \item 分配\texttt{transfer\_params\_t}结构体,包含传输类型(上下传等)、文件路径、偏移量等参数,转移文件锁
    \item 调用\texttt{pthread\_create}创建线程
    \item handler在传输启动前前发送150响应码,传输线程在后台执行实际I/O
    \item 线程函数\texttt{transfer\_thread\_func()}调用\texttt{transfer\_send\_file()}等对应版本执行传输
    \item 传输完成后线程设置\texttt{transfer\_result}并发送响应,最后清理数据连接和文件锁
    \item 客户端线程利用\texttt{pthread\_join()}等待线程结束,ABOR命令可以通过设置标志位、关闭数据传输来中止传输线程
\end{itemize}

\textbf{3. 被动模式竞态问题处理 (Thanks to Claude-Sonnet-4.5)}

被动模式下,服务器需要在监听套接字上\texttt{accept()}客户端连接。实现中遇到并解决了多个竞态问题:

\textbf{问题1:}高并发下(500个)\texttt{select()}显示可读但\texttt{accept()}返回\texttt{EWOULDBLOCK/EAGAIN}

\textbf{原因:}\texttt{select()}和\texttt{accept()}之间存在时间窗口,期间连接可能被撤销或被其他进程接受

\textbf{解决方案:}
\begin{itemize}[itemsep=0pt]
    \item 将监听套接字设为非阻塞模式(\texttt{net\_set\_nonblocking()})
    \item 实现重试机制:\texttt{accept()}失败时检测是否为\texttt{would\_block}错误,若是则短暂休眠(10ms)后重试,最多重试3次
    \item 每次重试前检查会话状态(\texttt{should\_quit})和套接字是否仍然有效,避免在已销毁的会话上操作
\end{itemize}

\textbf{问题2:}会话销毁时主线程阻塞在\texttt{select()}中

\textbf{原因:}\texttt{select()}在等待连接时持有会话锁会导致死锁,但释放锁后无法安全检测套接字是否被关闭

\textbf{解决方案:}
\begin{itemize}[itemsep=0pt]
    \item 在调用\texttt{select()}前释放\texttt{pthread\_mutex}锁,允许其他线程(如会话清理)继续执行
    \item 保存监听套接字的文件描述符\texttt{listen\_sock},\texttt{select()}返回后重新获取锁
    \item 验证\texttt{session->data\_listen\_socket}是否仍等于保存的\texttt{listen\_sock},若不同说明套接字已被关闭/重建,立即中止
    \item 当套接字被关闭时,\texttt{select()}会返回错误,正常处理错误并返回即可
\end{itemize}

\textbf{4. 文件锁}

\texttt{filelock.c}使用链表维护全局锁表\texttt{g\_file\_lock\_entries},每个节点\texttt{file\_lock\_entry\_t}包含:
\begin{itemize}[itemsep=0pt]
    \item \texttt{path}:文件路径;\texttt{readers/writers}:当前持有读/写锁的数量;\texttt{waiting\_writers}:等待写锁的数量
    \item \texttt{pthread\_cond\_t cond}:条件变量实现锁等待队列
    \item \texttt{file\_lock\_acquire\_shared()}:获得共享锁,等待无排它锁定,有非阻塞版本
    \item \texttt{file\_lock\_acquire\_exclusive()}:获得排它锁,等待无其他锁定,同样由有非阻塞版本
    \item 释放锁时利用\texttt{pthread\_cond\_broadcast()}唤醒所有等待线程;若没有线程持有锁则移除节点
\end{itemize}

\textbf{5. ABOR中断机制}

\begin{itemize}[itemsep=0pt]
    \item 控制连接启用OOB inline模式,ABOR命令可作为紧急数据发送
    \item 主线程检测到ABOR后设置\texttt{session->transfer\_should\_abort = 1}
    \item 传输线程在每次I/O循环前调用\texttt{session\_should\_abort\_transfer()}检查标志位
    \item 若需中断,线程设置\texttt{transfer\_result = TRANSFER\_STATUS\_ABORTED}并跳出循环
    \item 主线程可调用\texttt{session\_close\_data\_connection()}关闭数据套接字,使阻塞I/O立即返回
    \item 传输结束时根据\texttt{transfer\_result}发送对应响应码,如果传输正在执行,handler提前发送426响应码
\end{itemize}

\subsection{客户端核心实现}

\textbf{1. 连接管理}

\texttt{ControlConnection}继承\texttt{BaseConnection},使用\texttt{threading.RLock}保护接收操作:
\begin{itemize}[itemsep=0pt]
    \item \texttt{recv\_line()}:逐字节接收直到遇到CRLF(\texttt{\textbackslash r\textbackslash n})
    \item \texttt{recv\_multiline()}:检测首行是否为多行响应(码-消息),循环接收直到遇到"码<空格>"结束标记
    \item \texttt{DataConnection}支持主被动模式:\texttt{setup\_passive()}连接到服务器指定的地址;\texttt{setup\_active()}创建监听套接字并通过\texttt{accept()}等待服务器连接
\end{itemize}

\textbf{2. 命令注册与Handler模式}

\texttt{CommandRegistry}维护\texttt{commands}字典(命令名$\rightarrow$Handler类):
\begin{itemize}[itemsep=0pt]
    \item 注册:\texttt{register(command\_name, handler\_class)}将Handler类关联到命令名
    \item 执行:\texttt{execute(command, *args, **kwargs)}实例化对应Handler并调用\texttt{execute()}方法
    \item 这样新增命令只需定义继承\texttt{CommandHandler}的子类,实现\texttt{execute()},无需修改核心逻辑
\end{itemize}

\textbf{3. 传输管理器}

\texttt{TransferManager}使用\texttt{threading}管理异步传输:
\begin{itemize}[itemsep=0pt]
    \item 维护\texttt{transfers}字典(transfer\_id$\rightarrow$Transfer对象),每个\texttt{Transfer}包含状态、进度、线程等信息
    \item \texttt{Transfer}对象使用\texttt{pause\_event}和\texttt{cancel\_event}实现暂停/取消功能
    \item 传输线程在循环中检查\texttt{pause\_event}(暂停则阻塞)和\texttt{cancel\_event}(取消则退出)
    \item 支持进度回调:\texttt{progress\_callback(bytes\_transferred, total\_size)}定期报告进度
    \item 因(目前)FTP单数据连接限制,\texttt{max\_concurrent}实际限制为1,避免数据连接冲突
\end{itemize}

\textbf{4. GUI实现与测试兼容}

通过在\texttt{main.py}中加入参数控制实现兼容,比如 \texttt{-g} 进入图形化界面,默认进入CLI模式。CLI模式下,不使用 \texttt{-v} 则进入测试兼容版本,避免交互提示。
GUI实现借助了 Claude-Sonnet-4.5 的帮助

\end{document}
